\subsubsection{Groundwater Inflow (Baseflow)}
	A major component of river flow is groundwater inflow or baseflow, as opposed to water contributed by runoff. In SWAT, baseflow is determined by the ALPHA\_BFLOW parameter and can be adjusted for each subbasin. Prior to calibration, an effort was made to regionalize (or customize according to geographic location) this variable to account for the wide variations in baseflow conditions across the WRB. A model was constructed relating baseflow to local watershed characteristics and then this model was used to predict ALPHA\_BFLOW for each SWAT subbasin in the WRB; creating a local value of ALPHA\_BFLOW based upon local conditions.\footnote{The code used to carry out these and the following tasks is available here: \url{https://github.com/dnrwaterqualitymodeling/wisconsinRiverTMDL/tree/master/groundWater/baseFlow}}
	
	In order to construct a model relating baseflow contribution to watershed characteristics it was necessary to obtain observed values of baseflow. The baseflow separation routine from [Arnold when?] was used to determine baseflow from daily streamflow data. The observed streamflow data were retrieved from the USGS National Water Information System (NWIS) on August 27, 2014. All monitoring stations with daily flow data in the state of Wisconsin were used.  According to [Arnold article about only using bflow for certain drainage size] the baseflow separation routine should only be used for streams between 50 and 1,000 km$^2$ of contributing area and so only monitoring sites in this range were used. 
	
	This algorithm requires continuous daily observations of streamflow for at least one year, from which it determines the baseflow contribution from the hydrograph.	After the observed data were downloaded they were processed to ensure that only contiguous periods of streamflow of at least a year were used in the routine. For this analysis gaps of up to nine days were allowed in the record and still considered contiguous. If a monitoring site had a gap, or gaps, of longer than nine days, it was split into separate records at the gaps and each part assessed as to whether it contained at least a year of data, thus it was possible for a monitoring site to have several periods of contiguous streamflow records. If a record spanned less than one year of data it was not use in the analysis.
	
	Each record of streamflow was analyzed using the SWAT Bflow filter. This algorithm produced several estimates of ALPHA\_BFLOW, the SWAT parameter for baseflow contribution, one for each pass of the filter; for every record, the third (smoothest) pass was used. For sites with multiple records, the ALPHA\_BFLOW values were averaged, weighting the values according the length of the record. The end result was an ALPHA\_BFLOW value for each monitoring station that satisfied the requirements for the Bflow filtering algorithm.  

	Data regarding the landscape characteristics of the watershed for each monitoring station were retrieved from the Wisconsin DNR's Wisconsin Hydrography Dataset Plus (WHDPlus) [CITE]. Additionally, the EPA's Ecoregion boundaries level III were also used as a categorical predictor. We thought many different watershed characteristics could potentially affect baseflow and assessed the correlation of these with ALPHA\_BFLOW. We made an effort to avoid overfitting and multicollinearity and so chose constructed several models using only three predictors in any one model, using the predictors that showed the best correlation with baseflow.  The best model used average slope of watershed, average permeability, and the EPA ecoregion boundaries. The ecoregion boundaries were used as a categorical variable with an interaction effect with average watershed slope. The interaction term with slope was meant to allow for the expression of the effect of slope on baseflow contribution in different regions in the WRB, e.g., different slope terms for the Driftless Area and the central sands. 
	
	[Table of model terms?]
	
	This model was used to predict the ALPHA\_BFLOW for every small watershed in the WHDPlus dataset, upon which the 338 SWAT subbasins are derived. An area weighted average of these small watersheds was taken for each SWAT subbasin to aggregate the ALPHA\_BFLOW predictions. These values were used to update the groundwater files for each subbasin.