\subsubsection{Baseflow Phosphorus}
	Reference phosphorus, also known as background phosphorus or potential phosphorus level, refers to the amount of phosphorus found in a waterbody that would be present in the absence of human activity. The baseflow component of background phosphorus is not be simulated by default in SWAT and so the parameter controlling soluble phosphorus in groundwater, GW\_SOLP, needs to be populated manually. 
	
	In the Wisconsin River Basin SWAT model, we used values of reference baseflow phosphorus from a USGS study of nutrient concentrations in wadeable streams in Wisconsin \citet{robertson_wadeable_2006}. In their study, they used a multiple linear regression equation to predict reference phosphorus in nutrient boundaries known as ``environmental phosphorus zones", which they use as a way of dividing the state into smaller, homogeneous regions. These zones were derived in an earlier study by \citet{robertson_phosphoruszones_2006}. 
	For each of the phosphorus zones, the percent land area in agricultural use and percent land in urban use was calculated along with the number of point sources. Using these variables, a multiple linear regression model predicting the log concentration of phosphorus was constructed. To represent the scenario where human impact is negligible, the values of the predictors of this model (all of which represent human impact) are set to zero. With the predictors set to zero, the predicted value of the model is the median phosphorus concentration when human impact is zero.	These values are what we are using in the SWAT model to as background phosphorus. The SWAT subbasins were overlaid over the phosphorus zones, and the predicted reference phosphorus was extracted to each subbasin. For those subbasins that spanned more than one phosphorus zone, the area weighted average of the reference phosphorus was taken. These reference phosphorus levels were input into SWAT using the groundwater soluble P parameter.
	
	[map of baseflow P]
	
	\citet{robertson_wadeable_2006} intend their background phosphorus values to estimate median phosphorus concentration in streams when there is no human impact in the watershed. They do not specifically estimate the groundwater or baseflow contribution to reference phosphorus concentration. We believe that their median reference phosphorus estimate is still an accurate value of baseflow phosphorus concentration because a landscape under natural conditions (that is one without human impact) will experience very little runoff [cite TR-55].
	
	
	
	