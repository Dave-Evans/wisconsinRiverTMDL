\subsubsection{Internally Draining Areas}
	There is discussion of how the wetlands/internally drained areas are configured for the WR TMDL. Currently (6 Oct 2014), wetlands in the WRB were defined as being identified as wetlands by either the CDL or the Wisc Wetland Inventory. Wetlands were assumed to have a normal depth of 0.5 m, this was used for normal volume. The sink depths ($DEM_{filled} - DEM_{normal}$) were used to derive the maximum volume. The issue raised is how wetlands that have a very small difference between normal and maximum volume will affect the model. Currently there are many wetlands in the model that have a small difference between maximum and normal volumes, especially in riparian areas. The issue is (\emph{the writer, DE, is not of this view, so may not be capturing the argument accurately}) that this method will lead to a large proportion of the subbasin runoff being routed through  wetlands; or rather a larger proportion than is realistic, because SWAT does not look at where exactly the wetlands are, rather a lumped fraction of runoff is diverted through the wetlands. Additionally, this configuration might \textit{over}estimate wetland surface area, but \textit{under}estimate volume, leading to large areas of wetlands where the water will be routed to, but will immediately overflow.

	In comparing the new wetlands layer with the old, \href{run:T:/Projects/Wisconsin_River/Model_Documents/TMDL_report/figures/calibration_validation_figures/newVSold_wetlands/defaults_137.pdf}{(see here for the Baraboo river)} it can be seen there is not much difference between the two. Other reaches show that one method or the other is flashier, exhibiting higher peaks and lower low-flows, but as of now (10 Oct) there is no discernable difference. 

	The new method (7 Oct 2014) for calculating wetland parameters is simpler than previously and is based more on topography than actual or observed wetlands. The difference between a filled DEM and an unfilled DEM creates the sink depth. This serves as the basis for the wetlands. The herbaceous and woody wetlands and the cranberry codes in the CDL provide the wetland areas. \colorbox{yellow}{Review this!} These wetland areas are masked by the presence of landlocked ponds; thus if a wetland area (as defined by CDL) intersects with a land locked it is removed so as not to be double counted in both ponds and wetlands. The intersection of the wetland area defined by CDL and the filled area is the normal surface area and the normal volume is the sum of each pixel's sink depth times the cell resolution. Maximum surface area is the total sink (total contiguous filled) area (providing it intersects with a wetland code).
	
	\citet{almendinger_willowriverswat_2007} considered internally drained areas, wetlands (as id'd by WISCLAND) not connected to the main channel and lakes as ponds in their SWAT model. Wetlands, identified through WISCLAND, are considered SWAT wetlands only if they occur on the main channel. Similarly, \citet{kirsch_rock_2002} consider internally drained areas as wetlands if they overlapped with WISCLAND wetlands; if they did not, they were considered ponds. Additionally, other non-ID wetlands were identified by WISCLAND. \citet{almendinger_contructingsunrise_2010} modeled closed internal depressions as wetlands and open (those draining to the main channel) as ponds. This configuration closely matches the methodology we are currently (8 Oct) pursuing. \citet{freihoefer_mead_2007} seemed to exclude internally drained areas from their watershed delineation and it is not clear what was done with them after this. Wetlands were included as an HRU type, but there is no mention in their report as far was how the SWAT wetland routine was used.
	
	From SWAT theory (2009 edition): ``The algorithms used to model [ponds and wetlands] differ only in the options allowed for outflow calculation.'' Water balance for a pond or wetland: $V = V_{stored} + V_{flowIN} - V_{flowOUT} +V_{precip}-V_{evap}-V_{seep}$. The volume of water entering a pond or wetland is (partially) calculated as the amount of water from (basinwide) surface runoff, gw inflow and lateral flow times the fraction of area draining into the impoundment. "The volume of water entering the pond or wetland is subtracted from the surface runoff, lateral flow and groundwater loadings to the main channel." 

	Wetland outflow: if the volume of water stored in the wetland ($V$) is less than the normal volume ($V_{norm}$) then $V_{flowOUT}=0$. If pond volume is between normal volume and maximum volume $V_{flowOUT}=\frac{V-V_{norm}}{10}$. If the pond volume is greater than maximum volume, then $V_{flowOUT}=V-V_{max}$. 