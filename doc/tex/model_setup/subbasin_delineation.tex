\subsubsection{Subbasin Delineation}
To estimate sources of pollutants in a river, the first step is to delineate subbasins. The size of each subbasin is critical to water quality improvement following model development---they should be small enough that water quality improvement plans can address specific pollutant sources, while large enough that model results appropriately match the scale of model inputs and calibration data. Ultimately, the size of each subbasin depends on what the project is intending to achieve by simulating water quality, and therefore requires communication with water quality policy staff and watershed planners.

The Wisconsin River is a relatively large area for a TMDL project, and therefore much of the point and non-point load-reduction efforts will occur as nested projects within the overall TMDL framework. Each subbasin was scaled to a size that watershed managers and stakeholder groups can realistically account for and assess downstream improvements in water quality related to implementation of upstream best management practices.

In addition to the above guidelines, hydrologic and regulatory transitions were used to guide placement of subbasin transitions. TMDL subbasins were delineated: 
\begin{enumerate}
	\item to address specific water-quality impairments where local water quality does not meet codified standards; this is an EPA mandate for de-listing of water quality impairments. Consideration was given to streams that were estimated to be impaired, but where monitoring data does not exist to prove it.
	\item near point source outfalls; delineations were not required to be at precisely the location of the outfall, but rather close enough that flow could be accurately estimated by proportionally scaling modeled flow by upstream contributing area.
	\item at locations where water quantity and quality were measured during the model period for use in model calibration.
	\item at major transitions of water quality standards, for instance at river impoundments that receive lake criteria.
	\item at major hydrologic transitions such as the confluence of two large streams or where there are significant changes in landuse/landcover.
\end{enumerate}

After the locations of subbasin outfalls were identified, we delineated the contributing area upstream of each.
Rather than re-creating contributing areas from a DEM using ArcSWAT, we chose to aggregate contributing areas based on a previously existing watershed dataset (\href{ftp://dnrftp01.wi.gov/geodata/hydro_va_24k}{ftp://dnrftp01.wi.gov/geodata/hydro\_va\_24k}) based on the National Watershed Boundary Dataset (WBD) that honors the undeveloped geomorphology of the recently glaciated portion of the Wisconsin River basin.
We delineated 338 subbasins with an average size of 68 sq. km. ($\sigma$ = 80) where larger subbasins were located in areas with fewer water quality impairments and points sources.